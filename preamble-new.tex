\usepackage{amsmath,amsfonts,amsthm,amssymb,varioref,times, commath}
\usepackage{gensymb}
\usepackage{tikz}
\usepackage{textcomp}
\usepackage{hyperref}
\hypersetup{
 colorlinks=true,
 linkcolor=blue,
 filecolor=magenta, 
urlcolor=cyan,
}
\usepackage{lipsum}
%to resume numbering in a list
\usepackage{enumitem}
%----- arrows 
\usepackage{extarrows}

%    differential equatiosn 
\usepackage{diffcoeff}   %\diff[2]{x}{y}


%%%%%%pour ecrire en français avec les accents
\usepackage[utf8]{inputenc}
\usepackage[T1]{fontenc}
\usepackage{lmodern} % load a font with all the characters
\usepackage{units}
%%%%%%%Image-related packages
\usepackage{wrapfig}
\usepackage{float, graphicx}
\graphicspath{ {./img/} }
\usepackage{subcaption}
\usepackage[export]{adjustbox}

%%%%%%%pour faire des cadres
\usepackage{xcolor}
\usepackage{tcolorbox}
\usepackage{framed}
\usepackage[framemethode=tikz]{mdframed}


%%%%%%%chemistry frmulae
\usepackage{chemfig}
\usepackage{chemformula}
\usepackage[version=4]{mhchem}

% -------------- Circuits -------------------
\usepackage[european, straightvoltages]{circuitikz}

% Title & headers
\usepackage[explicit]{titlesec}
% Raised Rule Command:
% Arg 1 (Optional) - How high to raise the rule
% Arg 2 - Thickness of the rule
\newcommand{\raisedrulefill}[2][0ex]{\leaders\hbox{\rule[#1]{1pt}{#2}}\hfill}
\titleformat{\section}{\Large\bfseries}{\thesection. }{0em}{#1\,\raisedrulefill[0.4ex]{1pt}}

% pour ecrire sur +sieurs colonnes
\usepackage{multicol}
\setlength{\columnseprule}{0pt}
\setlength{\columnsep}{60pt}
% Fusion de lignes de tableaux.
\usepackage{multirow}
% Position verticale des lettres dans la ligne de tableau.
\usepackage{array}

% physics -----------------------------------------------------------
\newcommand{\To}{\longrightarrow}
\newcommand{\gpl}{\; g\cdot L^{-1}}
\newcommand{\gpmol}{\; g\cdot mol^{-1}}
\newcommand{\mpl}{\; mol\cdot L^{-1}}
\newcommand{\mps}{\; m\cdot s^{-1}}
\newcommand{\rps}{\; rad\cdot s^{-1}}
\newcommand{\kph}{\; km\cdot h^{-1}}
\newcommand{\mpss}{\; m\cdot s^{-2}}
\newcommand{\Dt}{\Delta t}
\newcommand{\vv}{\vec{v}}
\newcommand{\va}{\vec{a}}
\newcommand{\vp}{\vec{p}}
\newcommand{\vf}{\vec{F}}
\newcommand*{\Vf}[1]{\overrightarrow{F_\ensuremath{{#1}}}}
\newcommand{\es}[1]{\cdot10^{#1}}
\newcommand{\eng}[1]{\textcolor{purple}{(= #1})}
\usepackage{harpoon}
%\newcommand*{\vect}[1]{\overrightharp{\ensuremath{#1}}}
\newcommand*{\Vect}[1]{\overrightarrow{\ensuremath{#1}}}
\newcommand{\pfd}[1]{\sum \vec{F}_{ext_{#1}} &= \od{\vp_{#1}}{t} = m\cdot\va_{#1}}
\newcommand{\C}{\degree C}
\newcommand{\Delt}{\Delta t}

% --- Circuits ------------
\newcommand{\bipole}[1]{
\begin{circuitikz} \draw
(0,0) to[ #1 ] (2,0); 
\end{circuitikz} {\hspace{5mm}}}

% Chimie ---------------------------------
\newcommand{\oxo}{\ce{H3O+}_{(aq)}}
\newcommand{\eau}{\ce{H2O}_{(\ell)}}
\newcommand{\OH}{\ce{HO-}_{(aq)}}
\newcommand{\AH}{\ce{AH}_{(aq)}}
\newcommand{\A}{\ce{A-}_{(aq)}}
\newcommand{\MnO}{\ce{MnO_4^{-}}}
\newcommand{\conc}[1]{\left[{#1}\right]}
\newcommand{\couple}[2]{\ce{#1/#2}}


% Environnements ------------------------
\newcounter{exo}
\newenvironment{exo}[1][]
{\refstepcounter{exo} \begin{shaded}\noindent $\triangleright \quad$\textbf{Exercice~\theexo. #1} } { \end{shaded}}
\newenvironment{eg}
{\begin{shaded} \textbf{Exemple:} } { \end{shaded}}

\newenvironment{defn}[1]
{\begin{leftbar}\noindent \textbf{Définition :\textit{ \quad #1}} } { \end{leftbar}}

%\newenvironment{rmrq}
%{\begin{shaded} \textbf{Remarque.\quad } \itshape } { \end{shaded}}
\newenvironment{rmrq}
{\begin{mdframed}[backgroundcolor=blue!10, linewidth=0pt] \textbf{Remarque.\quad } \itshape } { \end{mdframed}}

\newenvironment{python}
{\begin{shaded} \textbf{A faire en PYTHON}\\ \itshape } { \end{shaded}}

% Shading colour -----------------------------
\definecolor{shadecolor}{gray}{0.9}

\date{}
\author{}

\renewcommand*\contentsname{Résumé}







